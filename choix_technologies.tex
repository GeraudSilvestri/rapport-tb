\section{Choix des technologies}
Ce chapitre présente les différentes technologies utilisées dans le cadre de ce travail. Il est divisé en plusieurs sections, chacune décrivant une technologie particulière.

Une mise en avant des possibilités considérées ainsi que la justification du choix final est présentée dans chaque section.

\subsection{Langage de programmation}
L'application peut être réalisée dans un nombre important de langages de programmation. Le choix de celui-ci est important car il détermine les technologies qui peuvent être utilisées pour le développement de l'application.
Les choix se sont rapidement restreints à deux langages : Java et C\#. Ces deux langages sont très similaires et sont tous deux orientés objet. Ils sont également tous deux multi-plateformes et disposent d'une large communauté.

La première itération de l'application était en C\# avec une interface graphique en WPF. Le fait que le C\# soit fortement incliné vers le développement Windows a été un frein à l'utilisation de cette technologie. En effet, le développement d'une interface graphique multi-plateforme en C\# est possible, mais nécessite l'utilisation de bibliothèques tierces. De plus, le développement d'une application multi-plateforme en C\# nécessite l'utilisation de Mono, une implémentation open-source de .Net. Cela rajouterait une couche de complexité supplémentaire et pas forcément nécessaire au projet.

Le choix s'est donc porté sur Java. Ce langage est multi-plateforme et dispose d'une large communauté. De plus, il est possible de développer une interface graphique en JavaFX, une bibliothèque graphique incluse dans le JDK. Cela permet de ne pas avoir à utiliser de bibliothèques tierces pour le développement de l'interface graphique. JavaFX permet la création d'interface graphique en FXML, un langage de balisage XML. Cela permet de séparer la logique de l'interface graphique de la logique métier. Cela permet également de faciliter la maintenance de l'application. De plus, il est possible d'utiliser SceneBuilder pour créer des interfaces graphiques de manière visuelle.

\subsection{Bibliothèque de programmation linéaire}
Le choix de la bibliothèque de programmation linéaire est important, car elle détermine les fonctionnalités disponibles pour la modélisation du problème. Le choix s'est restreint à deux bibliothèques : Gurobi, CPLEX and Google OR-Tools.

Gurobi et CPLEX sont des bibliothèques de programmation linéaire commerciales. Elles sont toutes deux très performantes et disposent d'une large communauté. Cependant, elles sont payantes et ne peuvent être utilisées gratuitement que pour des projets non commerciaux. De plus, elles ne sont pas open-source. Contrairement à Google OR-Tools qui est une bibliothèque open-source et gratuite.

Le choix s'est porté sur Gurobi, car elle est plus performante que Google OR-Tools. De plus, elle met à disposition une interface Java permettant de l'utiliser simplement dans une application Java. Google OR-Tools n'a pas été choisi, car elle ne met pas à disposition d'interface. Cela aurait nécessité l'utilisation d'une interface Java tierce, ce qui aurait rajouté une couche de complexité supplémentaire au projet.

Concernant la licence, Gurobi fournit des licenses étudiantes gratuites. Cela permet d'utiliser la bibliothèque sans besoin de payer dans le cadre de ce travail.