\documentclass[
    reds, % Saisir le nom de l'institut rattaché
    il, % Saisir le nom de l'orientation
]{heig-tb}

\usepackage[nooldvoltagedirection,european,americaninductors]{circuitikz}

\signature{mbernasconi.svg} % Remplacer par votre propre signature vectorielle.

\makenomenclature
\makenoidxglossaries
\makeindex

\addbibresource{bibliography.bib}

\usepackage{etoolbox}
\renewcommand\nomgroup[1]{%
  \item[\bfseries
  \ifstrequal{#1}{A}{Constantes physiques}{%
  \ifstrequal{#1}{B}{Groupes}{%
  \ifstrequal{#1}{C}{Autres Symboles}{}}}%
]}

\newcommand{\nomunit}[1]{%
\renewcommand{\nomentryend}{\hspace*{\fill}#1}}

\nomenclature[A, 02]{\(c\)}{\href{https://physics.nist.gov/cgi-bin/cuu/Value?c}
{Vitesse de la lumière dans le vide}
\nomunit{\SI{299792458}{\meter\per\second}}}

\nomenclature[A, 03]{\(h\)}{\href{https://physics.nist.gov/cgi-bin/cuu/Value?h}
{Constante de Planck}
\nomunit{\SI[group-digits=false]{6.62607015e-34}{\joule\per\hertz}}}

\nomenclature[A, 01]{\(G\)}{\href{https://physics.nist.gov/cgi-bin/cuu/Value?bg}
{Constante de gravitation universelle}
\nomunit{\SI[group-digits=false]{6.67430e-11}{\meter\cubed\per\kilogram\per\second\squared}}}

\nomenclature[B, 03]{\(\mathbb{R}\)}{Nombres réels}
\nomenclature[B, 02]{\(\mathbb{C}\)}{Nombres complexes}
\nomenclature[B, 01]{\(\mathbb{H}\)}{Quaternions}

\nomenclature[C]{\(V\)}{Volume constant}
\nomenclature[C]{\(\rho\)}{Indice de frottement sec}

\newacronym{gcd}{GCD}{Plus grand diviseur commun}
\newacronym{lcm}{LCM}{Plus petit multiple commun}

\newglossaryentry{heig-vd}{
    name=HEIG-VD,
    description={Haute École d'Ingénierie et de Gestion du canton de Vaud}
}
\newglossaryentry{hes-so}{
    name=HES-SO,
    description={Haute École Supérieure de Suisse Occidentale}
}
\newglossaryentry{latex}{
    name=latex,
    description={Un langage et un système de composition de documents}
}
\newglossaryentry{maths}{
    name=mathematics,
    description={Les mathematiques sont ce que les mathématiciens fonts}
}
% Auteur du document (étudiant-e) en projet de Bachelor
\author{Géraud Silvestri}

% Activer l'option pour l'accord du féminin dans le texte
\genre{male}

% Titre de votre travail de Bachelor
\title{Génération d'horaires automatique}

% Le sous titre est optionnel
\subtitle{Travail de Bachelor}

% Nom du professeur responsable
\teacher {Prof. R. Efstratios (HEIG-VD)}

% Mettre à jour avec la date de rendu du travail
\date{\today}

% Numéro de TB
\thesis{7212}



\surroundwithmdframed{minted}

%% Début du document
\begin{document}
\selectlanguage{french}
\maketitle
\frontmatter
\clearemptydoublepage

%% Requis par les dispositions générales des travaux de Bachelor
\preamble
\authentification

%% Résumé / Résumé publiable / Version abrégée
\begin{abstract}
    

\end{abstract}

%% Sommaire et tables
\clearemptydoublepage
{
    \tableofcontents
    \let\cleardoublepage\clearpage
    \listoffigures
    \let\cleardoublepage\clearpage
    \listoftables
    \let\cleardoublepage\clearpage
    \listoflistings
}

\printnomenclature
\clearemptydoublepage
\pagenumbering{arabic}

%% Contenu
\mainmatter

\chapter{Introduction}
\section{Contexte}
La problématique de la planification et optimisation automatisée des événements (automated timetabling, ressource scheduling) est très importante étant donné que le besoin de créer des horaires est présent dans de multiples domaines tels que scolaire ou médical.

L'élaboration d'un horaire peut être une tâche fastidieuse et chronophage pour les personnes en étant responsables. Ils doivent prendre en compte toutes les contraintes tel que les différents types de salle, la disponibilité des intervenants, les préférences personnelles, les contraintes de temps, etc. Cela peut prendre énormément de temps et est prône aux erreurs humaines.
\section{Cahier des charges}
\subsection{Objectif}
Le but de ce travail de bachelor est d'utiliser la génération d'horaires automatique afin d'offrir une solution aux problèmes mentionnés précédemment. Via des algorithmes sophistiqués et des techniques d’optimisation, il est possible de créer des horaires précis et équitable rapidement, réduisant ainsi le temps requis et le risque d’erreurs humaines lors de la création d’un horaire. De plus, l’automatisation apporte une flexibilité accrue, permettant de prendre en compte les préférences des gens, telles que les jours de congés spécifiques ou les heures de travail voulues pour ensuite générer un horaire satisfaisant pour tous les intervenants.

\subsection{Spécificités}
La planification concerne le placement de chaque événement sur une grille horaire (prédéfinie) et l’affectation des ressources (par exemple, salles et/ou intervenants).

Un modèle de recherche opérationnelle sera créé pour modéliser le problème de planification. Il utilisera des inégalités linéaires pour modéliser les contraintes et objectifs du problème.

\subsection{Fonctionnalités}

Afin de clarifier ce qui est attendu de ce travail, voici une liste des fonctionnalités que l'application devra avoir:

\begin{itemize}
    \item Il doit être possible de récupérer les données via un fichier JSON, de plus l'utilisateur doit pouvoir ajouter de nouvelles données via l'interface et les exporter au format JSON
    \item L'utilisateur doit pouvoir choisir les contraintes qu'il veut appliquer à l'horaire généré parmi une liste de contraintes mises à disposition
    \item L'utilisateur doit avoir une représentation graphique de l'horaire généré ainsi que la possibilité d'exporter celui-ci
    \item Dans le cas où un assortiment de contraintes rend l'horaire impossible, une analyse des raisons doit être fournie
\end{itemize}

\subsection{Livrables}
Les livrables attendus pour ce travail sont les suivants:

\begin{itemize}
    \item Un rapport détaillant les choix de conception et d'implémentation ainsi que les résultats obtenus
    \item Une application portable (Windows, Linux, MacOS) permettant de générer un horaire à partir de données JSON
    \item Un rapport intermédiaire présentant l'avancement du travail
    \item Un résumé publiable
    \item Un poster présentant le travail
\end{itemize}

\chapter{Choix des technologies}
\section{Choix des technologies}
Ce chapitre présente les différentes technologies utilisées dans le cadre de ce travail. Il est divisé en plusieurs sections, chacune décrivant une technologie particulière.

Une mise en avant des possibilités considérées ainsi que la justification du choix final est présentée dans chaque section.

\subsection{Langage de programmation}
L'application peut être réalisée dans un nombre important de langages de programmation. Le choix de celui-ci est important, car il détermine les technologies qui peuvent être utilisées pour le développement de l'application.
Les choix se sont rapidement restreints à deux langages : Java et C\#. Ces deux langages sont très similaires et sont tous deux orientés objet. Ils sont également tous deux multi-plateformes et disposent d'une large communauté.

La première itération de l'application était en C\# avec une interface graphique en WPF. Le fait que le C\# soit fortement incliné vers le développement Windows a été un frein à l'utilisation de cette technologie. En effet, le développement d'une interface graphique multi-plateforme en C\# est possible, mais nécessite l'utilisation de bibliothèques tierces. De plus, le développement d'une application multi-plateforme en C\# nécessite l'utilisation de Mono, une implémentation open-source de .Net. Cela rajouterait une couche de complexité supplémentaire et pas forcément nécessaire au projet.

Le choix s'est donc porté sur Java. Ce langage est multi-plateforme et dispose d'une large communauté. De plus, il est possible de développer une interface graphique en JavaFX, une bibliothèque graphique incluse dans le JDK. Cela permet de ne pas avoir à utiliser de bibliothèques tierces pour le développement de l'interface graphique. JavaFX permet la création d'interface graphique en FXML, un langage de balisage XML. Cela permet de séparer la logique de l'interface graphique de la logique métier. Cela permet également de faciliter la maintenance de l'application. De plus, il est possible d'utiliser SceneBuilder pour créer des interfaces graphiques de manière visuelle.

\subsection{Bibliothèque de programmation linéaire}
Le choix de la bibliothèque de programmation linéaire est important, car elle détermine les fonctionnalités disponibles pour la modélisation du problème. Le choix s'est restreint à deux bibliothèques : Gurobi, CPLEX and Google OR-Tools.

Gurobi et CPLEX sont des bibliothèques de programmation linéaire commerciales. Elles sont toutes deux très performantes et disposent d'une large communauté. Cependant, elles sont payantes et ne peuvent être utilisées gratuitement que pour des projets non-commerciaux. De plus, elles ne sont pas open-source. Contrairement à Google OR-Tools qui est une bibliothèque open-source et gratuite.

Le choix s'est porté sur Gurobi, car elle est plus performante que Google OR-Tools. De plus, elle met à disposition une interface Java permettant de l'utiliser simplement dans une application Java. Google OR-Tools n'a pas été choisi, car elle ne met pas à disposition d'interface. Cela aurait nécessité l'utilisation d'une interface Java tierce, ce qui aurait rajouté une couche de complexité supplémentaire au projet.

Concernant la licence, Gurobi fournit des licences étudiantes gratuites. Cela permet d'utiliser la bibliothèque sans besoin de payer dans le cadre de ce travail.

\chapter{Conclusion}
%%if
Bien que non nécessaire dans un rapport de Bachelor, la discussion finale d'un projet résume les résultats obtenus et dresse une conclusion objective du projet. Un manager de société est souvent amené à lire de nombreux rapport, il ne s'intéresse généralement qu'à l'introduction au contexte de l'étude et à sa conclusion.

Si nécessaire, n'hésitez pas à scinder votre conclusion en deux parties : une conclusion technique et une conclusion personnelle.

Il est de coutume de signer la conclusion...
%%fi

\vfil
\hspace{8cm}\makeatletter\@author\makeatother\par
\hspace{8cm}\begin{minipage}{5cm}
    %%if
    % Place pour signature numérique
    \printsignature
    %%fi
\end{minipage}

\clearpage
\printbibliography

\appendix
\appendixpage
\addappheadtotoc

%%if
\chapter{Première annexe}

Les annexes n'ont pas un contenu \underline{normatif} mais \underline{descriptif}. Tout contenu annexé ne doit pas être nécessaire à la bonne compréhension du travail.

Les annexes contiennent généralement :

\begin{itemize}
    \item les dessins mécaniques (mises en plan);
    \item les schémas électriques détaillés;
    \item des photographies du projet;
    \item des scripts et des extraits de code source;
    \item des documents techniques \pex \emph{datasheet};
    \item des développements mathématiques.
\end{itemize}
\section{Sous section}
\lipsum[1]
%%fi

\let\cleardoublepage\clearpage
\backmatter

\label{glossaire}
\printnoidxglossary
\label{index}
\printindex

% Le colophon est le dernier élément d'un document qui contient des notes de l'auteur concernant la mise en page et l'édition du document : il est parfaitement optionnel.
%%if
\clearpage
\Large\textbf{Colophon :}\par\normalsize
\thispagestyle{empty}
La qualité de cet ouvrage repose que le moteur \LaTeX. La mise en page et le format sont inspirés d'ouvrages scientifiques tels que le modèle de thèse de l'EPFL et celui des publications O'Reilly.

Les diagrammes et les illustrations sont édités depuis l'outil en ligne draw.io. Certaines illustrations ont été reprises dans Adobe Illustrator. Les représentations 3D sont exportées de SolidWorks et certains graphiques sont générés à la volée depuis un code source Python.

L'auteur fictive de ce document \emph{Maria Bernasconi} est un nom emprunté, par amusement, aux spécimens publiés par Postfinance.

Ce document a été compilé avec XeLaTeX.

La famille de police de caractères utilisée est \emph{Computed Modern} créée par Donald Knuth avec son logiciel METAFONT.
\vfil
Le Colophon est le dernier élément d'un document qui contient des notes de l'auteur concernant la mise en page et l'édition du document : il est parfaitement optionnel.
%%fi

\end{document}
