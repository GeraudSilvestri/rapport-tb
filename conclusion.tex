\section{Conclusion}

Au niveau du résultat, le projet est réussi. Le but était de faire un générateur d'horaires pour des étudiants. L'application est fonctionnelle et permet de faire tout ce qui est attendu, malgré le fait que certaines de ces fonctionnalités ne soient pas aussi poussées que voulues.

Personnellement, je suis satisfait du résultat. J'ai pu apprendre beaucoup de choses, que ce soit au niveau des technologies utilisées, ou au niveau de la gestion de projet. J'ai pu découvrir de nouvelles technologies, et j'ai pu approfondir mes connaissances dans d'autres. J'ai aussi pu apprendre à gérer un projet de A à Z, et à gérer mon temps de travail.

Au niveau des équations linéaires, nous les avons vues de manière très basique lors de notre cursus. Nous avons vu comment les résoudre, ou même comment faire un système d'équations de base, mais rien d'aussi complexe que ce qui a été fait ici. Ça c'est révélé être un challenge qui a été très intéressant à relever et malgré les difficultés rencontrées au début, plus j'apprenais à comprendre le fonctionnement de ces équations, plus il était facile d'en faire de nouvelles pour d'autres cas.

\subsection{Remerciements}

Je tiens à remercier mon directeur de travail de Bachelor, M. Rappos Efstratios, pour le soutien et l'aide qu'il m'a apporté tout au long de ce travail. Son aide pour comprendre les différentes équations, ainsi que son expertise pour éviter que je parte dans la mauvaise voie se sont prouvés d'une grande aide.

Je tiens aussi à remercier mon entourage, que ça soit ma famille ou mes amis, ils ont été d'un grand soutien tout au long de ce travail. Ils ont su me motiver et me soutenir quand j'en avais besoin et sans eux, je n'aurais pas pu arriver à ce résultat.

\vfil
\hspace{8cm}\makeatletter\@author\makeatother\par
\hspace{8cm}\begin{minipage}{5cm}
    %%if
    % Place pour signature numérique
\printsignature
    %%fi
\end{minipage}