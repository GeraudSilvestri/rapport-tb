\section{Cahier des charges}
\subsection{Objectif}
Le but de ce travail de bachelor est d'utiliser la génération d'horaires automatique afin d'offrir une solution aux problèmes mentionnés précédemment. Via des algorithmes sophistiqués et des techniques d’optimisation, il est possible de créer des horaires précis et équitable rapidement, réduisant ainsi le temps requis et le risque d’erreurs humaines lors de la création d’un horaire. De plus, l’automatisation apporte une flexibilité accrue, permettant de prendre en compte les préférences des gens, telles que les jours de congés spécifiques ou les heures de travail voulues pour ensuite générer un horaire satisfaisant pour tous les intervenants.

\subsection{Spécificités}
La planification concerne le placement de chaque événement sur une grille horaire (prédéfinie) et l’affectation des ressources (par exemple, salles et/ou intervenants).

Un modèle de recherche opérationnelle sera créé pour modéliser le problème de planification. Il utilisera des inégalités linéaires pour modéliser les contraintes et objectifs du problème.

\subsection{Fonctionnalités}

Afin de clarifier ce qui est attendu de ce travail, voici une liste des fonctionnalités que l'application devra avoir:

\begin{itemize}
    \item Il doit être possible de récupérer les données via un fichier JSON, de plus l'utilisateur doit pouvoir ajouter de nouvelles données via l'interface et les exporter au format JSON
    \item L'utilisateur doit pouvoir choisir les contraintes qu'il veut appliquer à l'horaire généré parmi une liste de contraintes mises à disposition
    \item L'utilisateur doit avoir une représentation graphique de l'horaire généré ainsi que la possibilité d'exporter celui-ci
    \item Dans le cas où un assortiment de contraintes rend l'horaire impossible, une analyse des raisons doit être fournie
\end{itemize}

\subsection{Livrables}
Les livrables attendus pour ce travail sont les suivants:

\begin{itemize}
    \item Un rapport détaillant les choix de conception et d'implémentation ainsi que les résultats obtenus
    \item Une application portable (Windows, Linux, MacOS) permettant de générer un horaire à partir de données JSON
    \item Un rapport intermédiaire présentant l'avancement du travail
    \item Un résumé publiable
    \item Un poster présentant le travail
\end{itemize}